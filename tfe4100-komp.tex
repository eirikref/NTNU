\documentclass[norsk, a4paper, 12pt, twoside, titlepage]{article}
\usepackage{babel, alltt, amsmath, float, epsfig, enumerate,
  graphicx, longtable, subfigure, amsfonts, fancybox, boxedminipage,
  fancyheadings, pifont, calc}
\usepackage[utf8]{inputenc}
\usepackage[T1]{fontenc}
\usepackage{makeidx}
\usepackage{times}

\pagestyle{fancyplain}

\renewcommand{\sectionmark}[1]%
	     {\markboth{#1}{}}

\newcounter{local}\renewcommand{\labelenumi}
	   {\setcounter{local}{171+\value{enumi}}%
	     \ding{\value{local}}}

\lhead[\fancyplain{}{\bfseries\thepage}]%
      {\fancyplain{}{\bfseries\leftmark}}
\rhead[\fancyplain{}{\bfseries\leftmark}]%
      {\fancyplain{}{\bfseries\thepage}}
\cfoot{}

\title{TFE4100 Kretsteknikk\\Kompendium}
\author{Eirik Refsdal <eirikref@pvv.ntnu.no>}

\makeindex

\begin{document}
\maketitle

\tableofcontents

\newpage

\section{Introduksjon til elektriske kretser}
Ladninger er bipolare, dvs. at elektriske effekter beskrives vha. både
positive og negative ladninger.  Elektriske ladninger eksisterer i
diskrete kvantiteter, $k \times q$, hvor $k$ er et heltall og $q$ er
\index{Elementærladning}elementærladningen $\pm1.6022 \times 10^{-19} C$.

\index{Strøm}
\index{Strøm!definisjon}
\subsection{Strøm}
Strøm er definert som ladning per tidsenhet,

\begin{equation}
i = \frac{dq}{dt},
\end{equation}
hvor $i$ er strøm i ampere (A), $q$ er ladning i coulomb (C) og $t$ er
tid i sekunder (s).

Det er bestemt at positiv retning for strøm er den retningen som de
positive ladningene beveger seg.

\index{Kirchhoff!strømlov}
Det kan kun gå strøm hvis vi har en lukket krets, og Kirchhoffs
strømlov sier at summen av strømmer i kretsen er lik null.  Mer
presist sier den at summen av strømmer inn og ut av en node er lik
null,

\begin{equation}
\sum_{n=1}^{N}i_{n} = 0
\end{equation}



\subsection{Spenning}
\label{Spenning}
\index{Spenning}
\index{Spenning!definisjon}
Spenning er definert som energi per ladning,

\begin{equation}
v = \frac{dw}{dq},
\end{equation}
hvor $v$ er spenning i volt (V), $w$ er energi i joule (J) og $q$ er
ladning i coulomb (C).

Spenning er egentlig forskjell i potensiale mellom to punkter i en
krets og angir energien som trengs for å flytte en ladning fra det ene
punktet til det andre.

\index{Jord}
Spenningen i et punkt er forskjellen i potenisale mellom punktet selv
og et definert nullnivå i kretsen, også omtalt som ``jord''.

\index{Kirchhoff!spenningslov}
Kirchhoffs spenningslov sier at summen av spenninger i en krets er lik null,
\begin{equation}
\sum_{n=1}^{N}v_{n} = 0
\end{equation}


\subsection{Strøm- og spenningskilder}
Vi deler kilder langs to akser; en som skiller mellom strøm- og
spenningskilder, og en som deler mellom uavhengige og avhengige kilder.

\index{Kilde!strøm}
\subsubsection{Strømkilder}
En ideell strømkilde leverer en bestemt mengde strøm til kretsen den
er en del av.  Spenningen som genereres av kilden er avhengig av
kretsen den er en del av.

\index{Kilde!spenning}
\subsubsection{Spenningskilder}
En ideell spenningskilde leverer en bestemt spenning mellom
terminalene/endepunktene sine, uavhengig av mengden strøm som passer
gjennom den.  Mengden strøm som kilden leverer til kretsen avhenger av
kretsen selv.

\index{Kilde!uavhengig}
\subsubsection{Uavhengige kilder}
Uavhengige kilder leverer den angitte mengden strøm eller spenning
fullstendig uavhengig av kretsen de er en del av.

\index{Kilde!avhengig}
\subsubsection{Avhengige kilder}
Avhengige kilder opererer som en funksjon av en annen strøm eller
spenning i kretsen, og leverer feks. $A \times v_{x}$ volt, hvor
$v_{x}$ er en annen spenning i kretsen.

Vi har fire typer avhengige kilder; spenningskontrollerte
spenningskilder, strømkontrollerte spenningskilder,
spenningskontrollerte strømkilder og strømkontrollerte strømkilder.

\index{Kilder!praktiske}
\subsection{Praktiske strøm- og spenningskilder}
Praktiske strøm- og spenningskilder har en indre motstand som
begrenser den spenningen eller strømmen de kan levere til
terminallasten.

Gitt en spenningskilde som leverer spenningen $v_{s}$ til en krets med
last $R_{L}$, vil strømmen som spenningskilden leverer til kretsen
typisk måtte gå mot uendelig når $R_{L}$ går mot null, slik at
spenningskilden får opprettholdt spenningen den skal levere.

Dette er naturligvis ikke mulig i praksis, og er begrenset av den
interne motstanden.

En spenningskilde vil være seriekoblet med en minst mulig motstand,
slik at spenningsfallet over motstanden skal bli minst mulig, mens en
strømkilde vil være parallellkoblet med en størst mulig motstand,
slik at strømtapet skal bli minst mulig.

\index{Effekt}
\subsection{Effekt}
\index{Effekt!definisjon}
Effekt er definert som arbeid per tidsenhet,

\begin{equation}
p = \frac{dw}{dt} = \left(\frac{dw}{dq}\right) \left(\frac{dq}{dt}\right) = vi,
\end{equation}
hvor $p$ er effekt i watt (W), $w$ er energi i joule (J), $t$ er tid i
sekunder (s), $q$ er ladning i coulomb (C), $v$ er spenning i volt (V) og
$i$ er strøm i ampere (A).

I likhet med strøm og spenning kan effekt være både positiv og
negativ, som kan tolkes dithen at en kilde feks. kan \emph{tilføre} energi,
mens en last \emph{bruker} energi.

Ved hjelp av Ohms lov (avsnitt \ref{Ohms_lov}) er alle følgende
uttrykk gyldige for effekten som forbrukes av en motstand,

\begin{equation}
p = vi = i^{2}R = \frac{v^{2}}{R}
\end{equation}


\index{Passiv fortegnskonvensjon}
\subsection{Passiv fortegnskonvensjon}
Når referanseretningen for strømmen gjennom et kretselement er i samme
retning som referansespenningsfallet over elementet, skal positivt
fortegn brukes i uttrykk som relaterer spenningen til strømmen.
Ellers skal negativt fortegn brukes.

\index{i-v-karakteristikk}
\subsection{\emph{i-v}-karakteristikk}
En \emph{i-v}-karakteristikk gir en sammenheng mellom strøm og spenning for
et gitt kretselement ved forskjellige strømmer og spenninger, og
fremstilles gjerne grafisk.

Ikke bare gir dette en karakteristikk av elementet, men det gjør det
også lett å lese av strømmen gjennom elementet gitt en bestemt spenning
(og motsatt), samt å lese effektbruk rett ut av grafen/tabellen.


\subsection{Motstand og Ohms lov}
\label{Ohms_lov}
\index{Motstand}
Strøm som går gjennom ledere eller andre kretselementer møter en viss
motstand, som er bestemt av de elektriske egenskapene til materialet
elementet er laget av. Denne motstanden forårsaker at energi vil bli
brukt opp og ``forsvinne'' ut av kretsen i form av varme.

En ideell motstand er en enhet som har lineære motstandsegenskaper i
samsvar med Ohms lov,

\index{Ohms lov}
\begin{equation}
v = iR,
\end{equation}
hvor $v$ er spenning i volt (V), $i$ er strøm i ampere (A) og $R$ er
motstand i ohm ($\Omega$).


Et materiales motstand er avhengig av dets \index{Motstandsevne}
motstandsevne, $\rho$, som er den inverse av materialets
\index{Ledningsevne}ledningsevne, $\sigma$. Av dette har man også
definert \index{Konduktans}konduktans, som er den inverse av motstand,

\index{Konduktans!definisjon}
\begin{equation}
G = \frac{1}{R},
\end{equation}
hvor $G$ er konduktans i siemens (S) og $R$ er motstand i ohm ($\Omega$).

Følgelig kan også Ohms lov skrives om ved hjelp av konduktans,

\begin{equation}
i = Gv
\end{equation}

Det er også viktig å merke seg at Ohms lov kun er en empirisk
tilnærming til de fysiske egenskapene til elektrisk ledende
materialer, som typisk vil slå feil ved svært høye strømmer og/eller
spenninger.


\subsection{Åpne og kortsluttede kretser}
Formelt sett er en
\index{Kortslutning}\index{Krets!kortsluttet}kortslutning definert som
et kretselement hvor spenningen over er lik null, uavhengig av
strømmen som går gjennom.  Det vil i praksis si at motstanden til
kretselementet går mot null, noe som er tilfelle for feks. mange
ledninger og kabler.

Tilsvarende er en \index{Krets!åpen}åpen krets definert som et
kretselement hvor motstanden går mot uendelig.


\subsection{Seriekobling}
\index{Seriekobling}
To eller flere kretselementer sies å være seriekoblet dersom samme
strøm går gjennom hvert av dem.

For motstander i serie kan man regne ut en ekvivalensmotstand,
$R_{EQ}$, som er gitt ved,

\begin{equation}
R_{EQ} = \sum_{n=1}^{N}R_{n}
\end{equation}


\subsection{Parallellkobling}
\index{Parallellkobling}
To eller flere kretselementer sies å være parallellkoblet dersom samme
spenning ligger over hvert av dem.

For motstander i parallell kan man regne ut en ekvivalensmotstand,
$R_{EQ}$, som er gitt ved,

\begin{equation}
R_{eq} = \frac{1}{\frac{1}{R_{1}} + \frac{1}{R_{2}} + \cdots + \frac{1}{R_{N}}}
\end{equation}


\subsection{Spenningsdeler}
\index{Spenningsdeler}
\index{Spenning!deling}
Seriekoblede kretselementer vil ha samme strøm gjennom seg, men
naturligvis ikke samme spenning over.  Spenningen over hvert enkelt av
dem er gitt ut fra forholdet mellom dets egen motstand og den totale
motstanden i seriekoblingen på følgende vis,

\begin{equation}
v_{n} = \frac{R_{n}}{R_{1} + R_{2} + \cdots + R_{n} + \cdots + R_{N}} v_{S}
\end{equation}


\subsection{Strømdeler}
\index{Strømdeler}
\index{Strøm!deling}
Parallellkoblede kretselementer vil ha samme spenning over seg, men
ikke uten videre samme strøm gjennom.  Strømmen gjennom hvert enkelt
av dem er gitt ut fra forholdet mellom dets egen mostand og den totale
motstanden i parallellkoblingen på følgende vis,

\begin{equation}
i_{n} = \frac{\frac{1}{R_{n}}}{\frac{1}{R_{1}} + \frac{1}{R_{2}} + \cdots + \frac{1}{R_{n}} + \cdots + \frac{1}{R_{N}}} i_{S}
\end{equation}


\subsection{Wheatstone bridge}
\index{Wheatstone bridge}
Wheatstone bridge-kretsen brukes til nøyaktige målinger av
kretselementer med motstand mellom 1 $\Omega$ og 1 M$\Omega$, og
består av fire motstander arrangert som vist i figur (sett inn figur
her).

I utgangspunktet skal spenningsforskjellen mellom node a og b være
null, slik at kretsen er i balanse.  Hvis dette ikke er tilfelle er
spenningsforskjellen gitt ved følgende likning,

\begin{equation}
v_{ab} = v_{a} - v_{b} = v_{s} \left(\frac{R_{2}}{R_{1} + R_{2}} -
\frac{R_{x}}{R_{3} + R_{x}}\right)
\end{equation}

For en krets i balanse kan $R_{x}$ regnes ut vha. følgende formel

\begin{equation}
R_{x} = \frac{R_{2}}{R_{1}}R_{3}
\end{equation}


\subsection{Måleinstrumenter}
\index{Måleinstrumenter}
\subsubsection{Ohmmeter}
\index{Ohmmeter}
\index{Motstand!måling}
Et ohmmeter måler et kretselements motstand.  Denne målingen kan kun
foregå når kretselementet ikke er koblet til andre elementer.

\subsubsection{Amperemeter}
\index{Amperemeter}
\index{Strøm!måling}
Et amperemeter måler strømmen som går gjennom et kretselement.  Det er
viktig at amperemeteret har en intern motstand som er tilnærmet lik
null og at det kobles i serie med elementet man ønsker å måle strømmen
gjennom. En parallellkobling av et amperemeter vil fungere som en
kortslutning og forstyrre kretsen.

\subsubsection{Voltmeter}
\index{Voltmeter}
\index{Spenning!måling}
Et voltmeter måler spenningen som ligger over et kretselement.  Det er
viktig at voltmeteret har en intern motstand som går mot uendelig (det
er naturligvis ikke mulig i praksis, men kan virke sånn sett i forhold
til resten av kretsen), samt at det kobles i parallell med elemenet vi
ønsker å måle.

\subsection{Andre begreper i elektriske nettverk}
\subsubsection{Gren}
\index{Gren}
En gren er en vilkårlig del av en krets som har to
terminaler/endepunkter.  En gren kan inneholde et vilkårlig antall
kretselementer og forbinder i praksis to noder.

\subsubsection{Node}
\index{Node}
En node er et punkt i en krets hvor to eller flere grener er koblet
sammen.  Man skiller gjerne mellom trivielle noder, som er punkter
hvor kun to grener møtes, og vesentlige noder, som er punkter hvor mer
enn to grener møtes.

\subsubsection{Løkke/sløyfe}
\index{Løkke}
\index{Sløyfe}
En løkke/sløyfe er en hvilken som helst bane av grener gjennom kretsen.

\subsubsection{Maske}
\index{Maske}
En maske er en løkke/sløyfe som ikke inneholder andre løkker/sløyfer.

\subsubsection{Jord}
\index{Jord}
Som nevnt i punkt \ref{Spenning} er jord et valgt nullpunkt i kretsen,
som man bruker som referansepunkt i utregninger.

\newpage
\section{Teknikker for kretsanalyse}

\subsection{Nodespenningsmetoden}
\index{Nodespenningsmetoden}
Nodespenningsloven har følgende fire grunnprinsipper,
\begin{enumerate}
\item Velg en referansenode, vanligvis jord.
\item Definer de $n-1$ andre vesentlige nodene som de ukjente
  variablene $V_{a}$, $V_{b}$, osv.
\item Skriv opp Kirchhoffs strømlov for hver av de $n-1$ nodene, hvor
  strømmene - som oftest - er uttrykt som $\frac{V}{R}$, der $V$
  vanligvis blir spenningsforskjellen mellom de aktuelle nodene,
  feks. $V_{a} - V_{b}$.
  Strømmer inn i noden angis med positivt fortegn, mens strømmer ut
  regnes som negative.
\item Løs likningssettet.
\end{enumerate}

\subsubsection{Kretser med strømkilder}
\index{Nodespenningsmetoden!med strømkilder}
Kretser med strømkilder gjør nodespenningsmetoden enda enklere, siden
det bare er å sette verdiene for strømkildene rett inn.

\subsubsection{Kretser med spenningskilder}
\index{Nodespenningsmetoden!med spenningskilder}
Spenningskilder i kretsen gjør også utregningene lettere, siden de gir
ferdige tallsvar for opptil flere noder.

\subsection{Maskestrømsmetoden}
\index{Maskestrømsmetoden}
Maskestrømsmetoden har mange likhetstrekk med nodespenningsmetoden, og
kan forklares med følgende tre prinsipper,

\begin{enumerate}
\item Definer samtlige maskestrømmer konsekvent - nærmere bestemt med
  klokka.
\item Skriv opp Kirchhoffs spenningslov for hver enkelt maskestrøm,
  hvor hver spenning uttrykkes som maskestrøm ganget med motstand. For
  kretselementer som berøres av to maskestrømmer, skal man for hver
  maskestrøm føre opp et uttrykk av typen, $(i_{x} - i_{y}) \times R$,
  hvor $i_{x}$ er maskestrømmen til masken vi analyserer akkurat nå og
  $i_{y}$ er maskestrømmen til den andre masken som berører
  kretselementet.  Bare vær konsekvent, så går alt bra.
\item Løs likningssettet.
\end{enumerate}

\subsubsection{Kretser med spenningskilder}
\index{Maskestrømsmetoden!med spenningskilder}
Maskestrømsmetoden er ekstra effektiv for kretser som kun består av
spenningskilder, siden det også her bare er å sette spenningsverdiene
rett inn i uttrykkene for Kirchhoffs spenningslov.

\subsubsection{Kretser med strømkilder}
\index{Maskestrømsmetoden!med strømkilder}
Kretser med strømkilder byr ikke på spesielt store problemer enn andre
kretser, men husk å ha med som en ekstra ligning at $i_{x} - i_{y} = n
A$, hvor $i_{x}$ og $i_{y}$ er de to maskestrømmene og $n A$ er
strømkilden på $n$ ampere.


\subsection{Node- og maskeanalyse i kretser med avhengige kilder}
\index{Maskestrømsmetoden!med avhengige kilder}
\index{Nodespenningsmetoden!med avhengige kilder}
Ved avhengige kilder i kretsen kan vi bruke følgende fremgangsmåte,
\begin{enumerate}
\item Se på de avhengige kildene som uavhengige og sett opp
  nodespennings- eller maskestrømslikninger deretter.
\item Erstatt avhengigheten, feks. $\beta i_{x}$, med nodespennings-
  eller maskestrømslikningen som uttrykker variabelen, i dette
  eksempelet $i_{x}$.
\item Løs likningssettet.
\end{enumerate}

\subsection{Cramers metode}
\index{Cramers metode}
Et typisk likningssett man får ved bruk av nodespennings- eller
maskestrømsmetoden kan se ut som dette,

\begin{eqnarray}
21i_{1} - 9i_{2} - 12i_{3} & = & -33 \\
-3i_{1} + 6i_{2} - 2i_{3} & = & 3 \\
-8i_{1} - 4i_{2} + 22i_{3} & = & 50
\end{eqnarray}

Det er ingen stor heksekunst å løse likningssettet på tradisjonelt
vis, men det kan ofte være like raskt å bruke Cramers metode for å få
ut de tre ukjente.

Hver ukjent i likningssettet uttrykkes da som forholdet mellom to
determinanter,

\begin{equation}
x_{k} = \frac{N_{k}}{\Delta},
\end{equation}
hvor $x_{k}$ er den ukjente vi ønsker å finne, $N_{k}$ er
determinanten for ukjent variabel $k$ og $\Delta$ er den såkalte
karakteristiske determinanten.

$\Delta$ er lik for alle $x_{k}$ og vil i tilfellet over være,
\begin{equation}
\Delta = \begin{vmatrix}21 & -9 & -12 \\ -3 & 6 & -2 \\ -8 & -4 &
    22\end{vmatrix}
\end{equation}

Determinanten i telleren, $N_{k}$, lages ved å bytte ut kolonne $k$ i
$\Delta$ med kolonnen på høyre side av likhetstegnet i likningssettet.

I tilfellet over får vi følgende tre determinanter,

\begin{equation}
N_{1} = \begin{vmatrix}-33 & -9 & -12 \\ 3 & 6 & -2 \\ 50 & -4 &
    22\end{vmatrix}
\end{equation}

\begin{equation}
N_{2} = \begin{vmatrix}21 & -33 & -12 \\ -3 & 3 & -2 \\ -8 & 50 &
    22\end{vmatrix}
\end{equation}

\begin{equation}
N_{3} = \begin{vmatrix}21 & -9 & -33 \\ -3 & 6 & 3 \\ -8 & -4 &
    50\end{vmatrix}
\end{equation}


\subsection{Superposisjonsprinsippet}
\index{Superposisjonsprinsippet}
Superposisjonsprinsippet sier at i en lineær krets med $N$ kilder, er
hver grenstrøm og hver grenspenning summen av de $N$ strømmer og
spenninger man får ved å sette alle kilder unntatt én lik null og løse
kretsen.

Algoritmen for å bruke superposisjonsprinsippet blir altså,
\begin{enumerate}
\item Velg én og én av kildene i kretsen, og erstatt de resterende
  strømkildene med en åpen krets og de resterende spenningskildene med
  en kortslutning.
\item Regn ut hver grenstrøm og grenspenning i denne ``nye'' kretsen
  med bare én kilde.
\item Summer alle grenstrømmer og grenspenninger fra de forskjellige
  kretsene.
\end{enumerate}


\subsection{Norton- og Théveninekvivalente kretser}
\label{avs:norton- og théveninekvivalente kretser}
En krets som består av ideelle strøm- og spenningskilder og lineære
motstander kan, \emph{sett fra en lasts ståsted}, erstattes med en
enkel krets bestående kun av én motstand og enten en strømkilde
koblet i parallell eller en spenningskilde koblet i serie.

Den ekvivalente kretsen med strømkilde kalles Nortonekvivalent, mens
kretsen med spenningskilde kalles Théveninekvivalent.

Det er essensielt å huske at dette er fra en lasts ståsted.

\subsubsection{Ekvivalensmotstand}
\index{Norton!-motstand}
\index{Thévenin!-motstand}
Den ekvivalente motstanden, som er den samme enten vi har en Norton-
eller en Théveninekvivalent krets, regnes ut ved følgende algoritme,

\begin{enumerate}
\item Fjern lasten fra kretsen.
\item Alle kilder i kretsen settes lik null, det vil igjen si at
  strømkilder erstattes med en åpen krets og spenningskilder med en
  kortslutning.
\item Beregn den totale motstanden mellom terminalene til lasten,
  fremdeles med selve lasten fjernet. Her er det en fordel å begynne
  på motstatt side av lasten i kretsen og arbeide seg systematisk mot
  lastterminalene.
\end{enumerate}

\subsubsection{Théveninspenning}
\index{Thévenin!-spenning}
Théveninspenningen finnes ved å følge følgende steg,
\begin{enumerate}
\item Fjern lasten og la lastterminalene stå igjen som en åpen krets.
\item Definer spenningen, $v_{OC}$ (open circuit), mellom
  lastterminalene.
\item Bruk ønsket teknikk for kretsanalyse (nodespenning, maskestrøm,
  osv.) til å beregne $v_{OC}$.
\item Théveninspenningen er lik spenningen mellom lastterminalene,
  $v_{T} = v_{OC}$.
\end{enumerate}

\subsubsection{Nortonstrøm}
\index{Norton!-strøm}
Nortonstrømmen finnes på følgende måte,
\begin{enumerate}
\item Erstatt lasten med en kortslutning.
\item Definer strømmen gjennom kortslutningen som Nortonstrømmen,
  $i_{N} = i_{SC}$ (short circuit).
\item Bruk ønsket teknikk for kretsanalyse for å finne $i_{SC}$.
\end{enumerate}

\subsection{Kildetransformasjon}
\index{Kildetransformasjon}
I avsnitt \ref{avs:norton- og théveninekvivalente kretser} ble det
slått fast at en krets kan erstattes med motstand og \emph{enten} en
strømilde koblet i parallell \emph{eller} en spenningskilde koblet i
serie. Med andre ord er også en krets bestående av den nevnte strømkilden i
parallell med motstanden ekvivalent med en krets bestående av
spenningskilden i serie med motstanden, så lenge vi benytter følgende
relasjon,

\begin{equation}
v_{T} = R_{T}i_{N}
\end{equation}

Ved å utnytte dette konseptet systematisk kan man lett redusere store
og tilsynelatende uoverkommelige kretser ned til en enkel og
håndterlig krets.

Altså: \emph{Spenningskilde i serie med motstand erstattes med
  strømkilde i parallell med motstand}

\subsection{Maksimal effektoverføring}
Teoremet for maksimal effektoverføring forteller oss hvor mye effekt
vi maksimalt kan overføre til lasten fra resten av kretsen.

Utledning og løsing av maksverdiproblemet gir oss maksimal effekt når
lasten er lik ekvivalensmotstanden, $R_{T} = R_{L}$.


\subsection{Grafisk analyse av ikke-lineære kretser}
Grafisk analyse av ikke-lineære kretser er et hendig knep for å finne
strøm gjennom og spenning over et ikke-lineært kretselement uten å
måtte gjøre grisete utregninger.

Trikset er å 
\begin{enumerate}
\item Sette det ikke-lineære kretselementet som last
\item Finne Thévenin- og Norton-ekvivalentene for kretsen
\item Plotte kretselementets $i-v$-karakteristikk sammen med en såkalt
  lastlinje, som har $V_{T}$ som eneste punkt på x-aksen og
  $\frac{V_{T}}{R_{T}}$ som eneste punkt på y-aksen.
\end{enumerate}

Det ikke-lineære kretselementets strøm og spenning er nå verdiene der
de to linjene møtes, det så kalte operasjonspunktet.


\section{Vekselsstrømanalyse}

\subsection{Kondensatorer}
En kondensator er egentlig en åpen krets i form av to plater med
``luft'' mellom.  Den burde altså ikke kunne føre strøm, men takket
være kvantemekaniske viderverdigheter (som ligger utenfor pensum) gjør
den nå engang det; men vel å merke kun ved endringer i spenning eller
strøm i kretsen.

Når kondensatoren har kommet i likevekt fungerer den nok en gang som
en åpen krets.  


\newpage
\printindex

\end{document}
