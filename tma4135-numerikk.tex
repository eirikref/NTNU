\documentclass[norsk, a4paper, 12pt, titlepage]{article}
\usepackage{babel, alltt, amsmath, float, epsfig, enumerate, graphicx,
  longtable, amsfonts, pifont, parskip}
\usepackage[utf8]{inputenc}
\usepackage[T1]{fontenc}
\usepackage{makeidx}
\usepackage{times}

\title{TMA4135 Matematikk 4D\\Kompendium i numerikk}
\author{Eirik Refsdal}

\makeindex

\begin{document}
\maketitle

\newpage
En mangel ved dagens ``autorative'' kompendium i matematikk 4, er at
numerikkbiten i matematikk 4D er fullstendig utelatt.  Dette er ikke
et forsøk på å lage et dyptpløyende og fantastisk kompendium, men er
kun ment som en oppsummering av de metodene i Kreyszigs
\textit{Advanced Engineering Mathematics} (8th Ed.) som er pensum i
TMA4135 Matematikk 4D.  Samt at notatet om Newtons metode for systemer
er tatt med, og forsøkt gjort lesbart.

Merk også at det er skrevet fint lite om konvergenshastigheter og
feilvurderinger her.  Det får bli en annen gang :)

Dette dokumentet er tilgjengelig i PDF- og LaTeX-format fra \newline
http://www.refsdal.no/eirik/dokumenter/


\newpage
\tableofcontents

\newpage


\section{Numerical Methods in General [K17]}

\subsection{Introduction [K17.1]}
Det eneste interessante som står her er egentlig avsnittet om
feilverdier ved numeriske metoder, som egentlig er opplagt.

\begin{equation}
a = \tilde{a} + \varepsilon, \text{True value = Approximation +
Error.}
\end{equation}

De nevner også den \textbf{relative feilen $\varepsilon_{r}$}, som er
definert ved

\begin{equation}
\varepsilon_{r} = \frac{\varepsilon}{a} = \frac{a - \tilde{a}}{a} =
\frac{\text{Error}}{\text{True value}}
\end{equation}

Avslutningsvis poengterer de at enhver numerisk metode bør oppgis
sammen med et feilestimat.


\subsection{Solution of Equations by Iteration [K17.2]}

Konseptet med metodene i dette kapittelet er at vi har en likning
(typisk $f(x) = 0$), kjører gjennom metoden med en gjettet inputverdi
$x_{0}$, bruker den utregnede verdien som ny inputverdi og gjentar
denne prosedyren (naturligvis med unntak av å gjette inputverdi, siden
vi i alle steg unntatt steg én bruker den utregnede verdien fra
forrige iterasjon) til vi har fått et akseptabelt svar.


\subsubsection{Fixed-Point Iteration for Solving Equations $f(x) = 0$}

\begin{equation}
x_{n+1} = g(x_{n})
\end{equation}

\textbf{Eks: K17.2.E1:} An interation process, $f(x) = x^{2} - 3x + 1
= 0$.
Det første vi må gjøre her er å skrive om likningen til å ha bare $x$
på venstre side.  Vi ser at dette kan gjøres ved å dele enten på $x$
eller 3.  Under vises utregning med deling på 3.

\begin{eqnarray*}
x & = & \frac{1}{3}(x^{2} + 1) \\
x_{n+1} & = & \frac{1}{3}(x_{n}^{2} + 1)
\end{eqnarray*}

Nå er det bare å sette inn initialverdien for $x$ og kjøre det ønskede
antall iterasjoner.


\subsubsection{Newton's Method for Solving Equations $f(x) = 0$}
\begin{equation}
x_{n+1} = x_{n} - \frac{f(x_{n})}{f'(x_{n})}
\end{equation}

Også kjent som Newton-Raphsons metode.  Hele poenget her er å ta
uttrykket du måtte få oppgitt, skrive det om til $f(x) = 0$ på et
eller annet vis, derivere det og bare fylle inn i formelen over.

\textbf{Eks: K17.2.11:} Design a Newton iteration for cube roots and
compute $\sqrt[3]{7}$.
\begin{eqnarray*}
x & = & \sqrt[3]{7} \\
x^{3} & = & 7 \\
f(x) & = & x^{3} - 7 = 0
\end{eqnarray*}

Her har vi nå en fin $f(x)$ som vi kan derivere og få $f'(x) =
3x^{2}$. Så er det bare å sette inn i formelen og vips er vi i mål.

\subsubsection{Secant Method for Solving Equations $f(x) = 0$}
\begin{equation}
x_{n+1} = x_{n} - f(x_{n}) \frac{x_{n} - x_{n-1}}{f(x_{n}) -
f(x_{n-1})}
\end{equation}

Sekantmetoden er mer eller mindre lik Newtons metode, men den
deriverte i uttrykket er, som vi ser av formelen over, byttet ut med
et uttrykk for sekanten mellom punktene ($x_{n}, f(x_{n})$) og
($x_{n-1}, f(x_{n-1})$).


\subsection{Interpolation [K17.3]}
Målet med de følgende interpolasjonsmetodene er å finne et polynom,
$p_{n}$, som passer inn med et sett gitte punktet.  Motivasjonen for å
finne et polynom er at dette er kontinuerlig og fint og både kan
deriveres og integreres.

\subsubsection{Lagrange Interpolation}

\begin{equation}
f(x) = p_{n}(x) = \sum_{k=0}^{n}L_{k}(x)f_{k} =
\sum_{k=0}^{n}\frac{l_{k}(x)}{l_{k}(X_{k})}f_{k}
\end{equation}

Over er den generelle formelen for Lagrangeinterpolasjon, og under
følger måten man faktisk må sette det opp på når man skal gjøre
oppgaver.

\begin{eqnarray*}
p_{1}(x) & = & L_{0}(x)f_{0} + L_{1}(x)f_{1} \\
& = & \frac{x - x_{1}}{x_{0} - x_{1}} \times f_{0} + \frac{x -
x_{0}}{x_{1} - x_{0}} \times f_{1}
\end{eqnarray*}

For $p_{1}$ er det to ledd i uttrykket (og altså to $L$-uttrykk,
$L_{0}$ og $L_{1}$), for $p_{2}$ er det tre, osv.  Disse
$L$-uttrykkene settes opp som følger:

\begin{itemize}
\item Hvis det er 2 $L$-uttrykk, skal det være ett ``ledd'' i teller
og ett i nevner, hvis det er 3 $L$-uttrykk, skal det være to ``ledd''
i teller og to i nevner, osv.
\item Telleren skal være $\prod (x - x_{j})$ der $x_{j}$ er alle
verdier fra 0 til $n$, unntatt det tallet i rekket som vi lager en
$L$-verdi for akkurat nå.  Altså: Gitt at vi har tre ledd (0, 1 og 2),
så skal vi for $L_{1}$ ha telleren $(x - x_{0})(x - x_{2})$.  Vi
hopper over 1, men tar alle de andre.
\item Nevneren i den tilsvarende situasjonen vil være $(x_{1} -
x_{0})(x_{1} - x_{2})$.  Altså omtrent som telleren, men vi bruker
$x_{0}$ i stedet for bare $x$.
\end{itemize}

Se f.eks. eksempel 1 side 850 i Kreyzsig.

En ulempe med Lagrange er at vi må lage et helt nytt uttrykk så fort
vi skal kjøre en iterasjon til.


\subsubsection{Newton's Divided Difference Interpolation}

\begin{eqnarray*}
f(x) & \approx & f_{0} + (x - x_{0})f[x_{0}, x_{1}] + (x - x_{0})(x -
x_{1})f[x_{0}, x_{1}, x_{2}] + \cdots \\
& & + (x - x_{0})(x - x_{1})\cdots (x - x_{n-1}f[x_{0}, \cdots, x_{n}]
\end{eqnarray*}

Det viktige her (og for de to etterfølgende metodene) er å sette opp
en differansetabell (se tabellen under, som er hentet fra side 855 i
Kreyszig).  Poenget med denne tabellen er å sette inn oppgitte
$x$-verdier med tilhørende $f(x)$ (de to kolonnene lengst til
venstre).  Kolonne 3 er $f[x_{0}, x_{1}]$ fra uttrykket over, kolonne
4 er $f[x_{0}, x_{1}, x_{2}]$, osv.

Verdiene i kolonne 3 regnes ut ved å ta differansen av $f(x)$-verdiene
i kolonnen til venstre og dele på differansen av de tilhørende
$x$-verdiene.  Pass på at du tar rette $x$-verdier.  Det skal alltid
være ``ytterverdiene''.  I kolonne 5 skal du feks. ta differansen
mellom de to tallene fra kolonne 4, men dele på differansen mellom
11.0 og 8.0.

\begin{tabular}{rc|ccc}
\hline
$x_{j}$ & $f_{j} = f(x_{j})$ & $f[x_{j}, x_{j+1}]$ & $f[x_{j},
x_{j+1}, x_{j+2}]$ & $f[x_{j} \cdots x_{j+3}]$ \\
\hline

8.0  & 2.079 442 &           &            &           \\
     &           & 0.117 783 &            &           \\
9.0  & 2.197 225 &           & -0.006 433 &           \\
     &           & 0.108 134 &            & 0.000 411 \\
9.5  & 2.251 292 &           & -0.005 200 &           \\
     &           & 0.097 735 &            &           \\
11.0 & 2.397 895 &           &            &           \\
\hline
\end{tabular}

Så er det bare å sette verdiene inn i uttrykket over.

\begin{eqnarray*}
f(x) \approx p_{3}(x) & = & f_{0} + (x - x_{0})f[x_{0}, x_{1}] + (x -
x_{0})(x - x_{1})f[x_{0}, x_{1}, x_{2}] +\\
& & (x - x_{0})(x - x_{1})(x - x_{2})f[x_{0}, x_{1}, x_{2}, x_{3}] \\
& = & 2.079 442 + (x - 8.0)\times 0.117 783 + (x - 8.0)(x - 9.0)
\times\\
& & (-0.006 433) + (x - 8.0)(x - 9.0)(x - 9.5) \times 0.000 411
\end{eqnarray*}

Så er det bare å sette inn verdien for $x$, feks. 9.2, og regne ut.

Det fine med denne metoden er at vi slipper å lage et helt nytt
uttrykk når vi skal kjøre en iterasjon til.  Vi kan bare legge til et
nytt ledd til det gamle uttrykket, og kjøre på.

I tillegg er det også verdt å merke seg at den takler tilfeldig
avstand mellom punktene, sånn som rekken 8.0, 9.0, 9.5, 11.0 over.

\subsubsection{Newton's Forward Difference Formula}
\begin{eqnarray*}
f(x) \approx p_{n}(x) & = & \sum_{s=0}^{n} \binom{r}{s}
\Delta^{s}f_{0} \\
& = & f_{0} + r\Delta f_{0} + \frac{r(r - 1)}{2!} + \cdots + \\
& & \frac{r(r - 1)\cdots (r - n + 1)}{n!}\Delta^{n}f_{0}
\end{eqnarray*}

For praktisk utregning er det viktig å merke seg at $r = \frac{x -
x_{0}}{h}$, hvor $x$ er verdien vi skal finne, $x_{0}$ er det første
punktet vi har i det oppgittet datasettet vårt og $h$ er steglengden
(avstanden mellom hvert datapunkt).

For utregning i praksis setter man opp en differansetabell og setter
inn tallene fra den i uttrykket.  Se side 857 i Kreyzsig for et
eksempel.  Husk for all del å trekke fra $1$ mer i $r$ i telleren for
hvert ledd.  De mystiske $\Delta^{s}$-uttrykkene er tallene fra
differansetabellen, som lett vist på side 857.

Denne metoden virker kun på datasett hvor det er lik avstand mellom
datapunktene.

\subsubsection{Newton's Backward Difference Formula}
Metoden er i grunnen helt lik den forrige, men i $r$-uttrykkene i
telleren legger vi til $1$ for hver gang i stedet for å trekke fra.
Se eksempel side 858.

\subsection{Numerical Integration and Differentiation [K17.5]}

\subsubsection{Rectangular Rule. Trapezoidal Rule}

Rectangular rule først:
\begin{equation}
J = \int_{a}^{b}f(x)dx \approx h[f(x_{1}*) + f(x_{2}*) + \cdots +
f(x_{n}*)]
\end{equation}

Hvor $h$ i uttrykket over er $\frac{b - a}{n}$.


Så trapezoidal rule:
\begin{equation}
J = \int_{a}^{b}f(x)dx \approx h[\frac{1}{2}f(a) + f(x_{1}) + f(x_{2})
+ \cdots + f(x_{n-1} + \frac{1}{2}f(b)]
\end{equation}

\subsubsection{Simpson's Rule of Integration}

\begin{equation}
\int_{a}^{b}f(x)dx \approx \frac{h}{3}(f_{0} + 4f_{1} + 2f_{2} +
4f_{3} + \cdots 2f_{2m-2} + 4f_{2m-1} + f_{2m})
\end{equation}

hvor $h = (b-a)/2m$ og $f_{j} = f(x_{j})$.

Denne er også oppgitt i Rottmann, side 174.


\subsubsection{Adaptive Integration}

Adaptiv integrering vil si at man prøver å tilpasse $h$ til
forandringen i $f(x)$; lave verdier av $h$ der forandringen i $f(x)$
er stor og store verdier av $h$ der forandringen i $f(x)$ er liten.
Konseptet kan brukes sammen med hvilken som helst metode for numerisk
integrasjon.  Se side 876 i Kreyszig for et eksempel.

\newpage
\section{Numerical Methods in Linear Algebra [K18]}
%\subsection{Linear Systems: Gauss Elimination [K18.1]}
%Det her er i grunnen bare teoretisk bakgrunnsstoff om hvordan vanlig
%Gauss-eliminasjon av matriser fungerer.

\subsection{Linear Systems: LU-Factorization, Matrix Inversion [K18.2]}
Cluet med de følgende metodene er å faktorisere en matrise $A$ til de
to matrisene $L$ og $U$, henholdsvis \textit{lower triangular} og
\textit{upper triangular}. 

Vi bruker her eksempel K18.2.1 (side 895-896) for å illustrere de
generelle poengene ved metodene.

\begin{equation*}
A = \left[ \begin{array}{ccc}
3 & 5 & 2 \\
0 & 8 & 2 \\
6 & 2 & 8 \end{array} \right]
= LU =
\left[ \begin{array}{ccc}
1 & 0 & 0 \\
m_{21} & 1 & 0 \\
m_{31} & m_{32} & 1 \end{array} \right]
\left[ \begin{array}{ccc}
u_{11} & u_{12} & u_{13} \\
0 & u_{22} & 2u_{23} \\
0 & 0 & u_{33} \end{array} \right]
\end{equation*}

Over ser vi rent visuelt hvor de to matrisene har fått navnene sine
fra. Videre kan vi også sette opp følgende uttrykk:

\begin{equation}
Ax = LUx = b
\end{equation}



\subsubsection{Doolittle's Method}
Denne metoden går ut på at vi tar det generelle grunnlaget fra forrige
avsnitt, og sier følgende:

\begin{equation}
Ly = b\text{, hvor } y = Ux
\end{equation}

Vi løser så først for $y$ og deretter for $x$.  Se resten av eksempel
K18.2.1 for en fullstendig gjennomgang av metoden.

\subsubsection{Crout's Method}
Denne metoden er helt like Doolittles metode, rent bortsett fra at det
er matrisen $U$ som har en diagonal bestående av kun enere.

\subsubsection{Cholesky's Method}
Choleskys metode er svært lik de to forrige metodene, men den lille
forskjellen her er at $U = L^{T}$.  Se Kreyszig side 897 for eksempler.

\subsection{Linear Systems: Solution by Iteration [K18.3]}
Metodene i forrige kapittel er såkalte \textit{direkte metoder}, som
krever en rekke utregninger før det til slutt kommer ut et svar.
Metodene i dette kapittelet er såkalte \textit{indirekte} eller
\textit{iterative metoder}, altså metoder hvor vi begynner med et
estimat og får mer og mer nøyaktige svar for hver runde med
utregnigner vi foretar.

Iterative metoder brukes hvis konvergenshastigheten er stor eller
systemene inneholder svært mange nuller.

\subsubsection{Gauss-Seidel Iteration Method}
Trikset i denne metoden er å skrive om et likningsett av typen

\begin{equation*}
\begin{array}{rcrcrcrcr}
x_{1}      & - & 0.25x_{2} & - & 0.25x_{3} &   &           & = & 50 \\
-0.25x_{1} & + & x_{2}     &   &           & - & 0.25x_{4} & = & 50 \\
-0.25x_{1} &   &           & + &     x_{3} & - & 0.25x_{4} & = & 25 \\
           & - & 0.25x_{2} & - & 0.25x_{3} & + &     x_{4} & = & 25 \\
\end{array}
\end{equation*}

til formen

\begin{equation*}
\begin{array}{rcrcrcrcrcr}
x_{1} &=&           && 0.25x_{2} &+& 0.25x_{3} & &           &+& 50 \\
x_{2} &=& 0.25x_{1} &&           & &           &+& 0.25x_{4} &+& 50 \\
x_{3} &=& 0.25x_{1} &&           & &           &+& 0.25x_{4} &+& 25 \\
x_{4} &=&           && 0.25x_{2} &+& 0.25x_{3} & &           & & 25
\end{array}
\end{equation*}

Deretter velger vi estimater for alle $x$-verdiene, feks. $x_{1}^{(0)}
= 100, x_{2}^{(0)} = 100, x_{3}^{(0)} = 100, x_{4}^{(0)} = 100$, og
begynner den iterative prosessen.

Først regner vi ut en ny $x_{1}$, $x_{1}^{(1)}$, setter inn denne for
$x_{1}$ i tabellen, regner ut en ny $x_{2}$, $x_{2}^{(2)}$, setter inn denne
i tabellen, osv.  Slik fortsetter vi til ønsket presisjon er oppnådd
eller antallet ønskede iterasjoner er gjennomført.


\subsubsection{Jacobi Iteration}
Jacobi-iterasjon er helt lik Gauss-Seidel-iterajon, med ett bittelite
unntak; ingen av de utregnede $x$-verdiene benyttes før man har
gjennomført en hel iterasjon (altså regnet ut samtlige $x$-verdier).

\newpage
\section{Numerical Methods for Differential Equations [K19]}
\subsection{Methods for First-Order Differentical Equations [K19.1]}
Disse metodene er såkalte ``steg for steg''-metoder, hvor vi først
regner ut $y_{0} = y(x_{0})$ og deretter fortsetter å regne ut verdier
for $y(x)$ ved $x$-punkter som følger:

\begin{eqnarray*}
x_{1} & = & x_{0} + h \\
x_{2} & = & x_{0} + 2h \\
x_{3} & = & x_{0} + 3h \\
      & \vdots & \\
x_{N} & = & x_{0} + Nh
\end{eqnarray*}

\subsubsection{Euler Method (Euler-Cauchy)}

\begin{equation}
y_{n+1} = y_{n} + hf(x_{n}, y_{n})
\end{equation}


\subsubsection{Improved Euler Method (Heun's Method)}
I denne metoden regner vi først ut hjelpeverdien $y_{n+1}^{*}$,

\begin{equation}
y_{n+1}^{*} = y_{n} + hf(x_{n}, y_{n})
\end{equation}

og deretter den nye verdien,

\begin{equation}
y_{n+1} = y_{n} + \frac{1}{2}h\left[f(x_{n}, y_{n}) + f(x_{n+1},
  y_{n+1}^{*})\right]
\end{equation}

\subsubsection{Runge-Kutta Methods}
Metoden gir en løsning på initialverdiproblemet $y' = f(x, y),
y(x_{0}) = y_{0}$ ved ekvidistante punkter, $x_{1} = x_{0} + h, x_{2}
= x_{0} + 2h, \cdots, x_{N} = x_{0} + Nh$.

Først regner vi ut fire hjelpeverdier, $k_{1}$, $k_{2}$, $k_{3}$ og
$k_{4}$,

\begin{eqnarray}
k_{1} & = & hf(x_{n}, y_{n}) \\
k_{2} & = & hf(x_{n} + \frac{1}{2}h, y_{n} + \frac{1}{2}k_{1}) \\
k_{3} & = & hf(x_{n} + \frac{1}{2}h, y_{n} + \frac{1}{2}k_{2}) \\
k_{4} & = & hf(x_{n} + h, y_{n} + k_{3}) \\
\end{eqnarray}

Deretter regner vi ut den nye verdien, $y_{n+1}$,

\begin{equation}
y_{n+1} = y_{n} + \frac{1}{6}(k_{1} + 2k_{2} + 2k_{3} + k_{4})
\end{equation}


\subsubsection{Runge-Kutta-Fehlberg}
RKF kan umulig bli gitt på eksamen med mindre de oppgir formlene,
simpelthen fordi det er alt for mange uinteressante koeffisienter å
holde rede på.  Ikke minst vil det være et sant helvete å skulle
gjennomføre metoden vha. en HP30S.


\subsection{Methods for Systems and Higher Order Equations [K19.3]}
%\subsubsection{Euler's Method for Systems}


\subsubsection{Runge-Kutta Methods for Systems}
Her brukes bare de samme formlene som i vanlig Runge-Kutta, men alt i
alt er det likevel litt mer grisete.  Se eksempel K19.3.2 for en full
gjennomgang av et faktisk regneeksempel.


%\subsubsection{Runge-Kutta-Nyström Methods}

%\subsection{Methods for Elliptic Partial Differential Equations [K19.4]}
%\subsubsection{Difference Equations for the Laplace and Poisson Equations}
%\subsubsection{Dirichlet Problem}
%\subsubsection{ADI Method}

%\subsection{Methods for Parabolic Equations [K19.6]}
%\subsubsection{Crank-Nicolson Method}

\newpage
\section{Andre metoder}
\subsection{Newtons metode for systemer av ikke-lineære likninger}
Newtons (eller Newton-Raphsons) metode er en numerisk metode for å
finne numeriske løsninger av et system av ligninger:

\begin{eqnarray*}
f_{1}(x_{1}, \dots , x_{n}) & = & 0 \\
f_{2}(x_{1}, \dots , x_{n}) & = & 0 \\
& \vdots & \\
f_{n}(x_{1}, \dots , x_{n}) & = & 0
\end{eqnarray*}

Dette er egentlig ikke en spesielt komplisert metode, men
pensumnotatet som prøver å beskrive metoden er ikke akkurat
definisjonen på forståelig.  Her følger derfor en mer ``rett på
sak''-fremgangsmåte for å faktisk løse oppgaver av denne typen.  Det
gjør vi like gjerne med et eksempel.

Ifølge faglærer har det hendt at det har blitt gitt systemer med tre
likninger til eksamen, så det er kanskje en fordel å om nødvendig
friske opp matrisekunnskapene sine til å dekke også denslags.

\textbf{Eks: Eksamen, august 2003, oppgave 7c}

Gjør én iterasjon med Newtons metode på det ikke-lineære
likningssystemet

\begin{eqnarray*}
10y_{1} - y_{2} - 20 & = & 0 \\
y_{1} + (9 + y_{1}^{2})y_{2} & = & 0
\end{eqnarray*}

Bruk $y_{1} = 2$ og $y_{2} = 0$ som startverdier.

\textbf{Løsning}

Først og fremst må man ha klart for seg hva vi er ute etter å finne,
og det er de neste verdiene av $y_{1}$ og $y_{2}$.  Vi kaller
initialverdiene for $y_{1,0}$ og $y_{2,0}$, og skal finne $y_{1,1} =
y_{1,0} + \Delta y_{1,0}$ og $y_{2,1} = y_{2,0} + \Delta y_{2,0}$.

Det første vi må gjøre er å sette opp en Jacobimatrise for
likningssettet, og den er på formen,

\begin{eqnarray*}
J = \left[
\begin{array}{ccc}
\frac{\partial f_{1}}{\partial x_{1}} & \cdots & \frac{\partial
  f_{1}}{\partial x_{n}} \\
\vdots & \ddots & \\
\frac{\partial f_{n}}{\partial x_{1}} & \cdots & \frac{\partial
  f_{n}}{\partial x_{n}}
\end{array}
\right]
\end{eqnarray*}

Som vi ser øker $x$-verdiene langs den horisontale aksen og f-verdiene
langs den vertikale.  Vi ser altså at det er snakk om å derivere
likning $f_{n}$ med hensyn på variabel $x_{n}$ for de forskjellige
cellene/elementene i matrisen.

I vårt tilfelle blir dette
\begin{eqnarray*}
J(y_{1,0}, y_{2,0}) = \left[
\begin{array}{cc}
10 & -1 \\
1 + 2y_{1}y_{2} & 9 + y_{1}^{2}
\end{array}
\right]
\end{eqnarray*}

For å finne $y_{1,1}$ og $y_{2,1}$ må vi så løse likningen
\begin{equation*}
J(y_{1,0}, y_{2,0})
\left(
\begin{array}{cc}
\Delta y_{1,0}\\
\Delta y_{2,0}
\end{array}
\right)
= - \left(
\begin{array}{cc}
10y_{1,0} - y_{2,0} -20 \\
y_{1,0} + (9 + y_{1,0}^{2})y_{2,0}
\end{array}
\right)
\end{equation*}
med hensyn på $(\Delta y_{1,0}\ \Delta y_{2,0})^{\text{T}}$,

Vi setter så inn initialverdiene våre, altså tallverdier for $y_{1,0}$
og $y_{2,0}$, og får
\begin{equation*}
\left(
\begin{array}{cc}
10 & -1 \\
1  & 13
\end{array}
\right)
\left(
\begin{array}{cc}
\Delta y_{1,0}\\
\Delta y_{2,0}
\end{array}
\right)
= -
\left(
\begin{array}{cc}
0\\
2
\end{array}
\right)
\end{equation*}


Dette gir videre
\begin{eqnarray*}
\left(
\begin{array}{cc}
\Delta y_{1,0}\\
\Delta y_{2,0}
\end{array}
\right)
& = & -
\left(
\begin{array}{cc}
10 & -1 \\
1  & 13
\end{array}
\right)^{-1}
\left(
\begin{array}{cc}
0\\
2
\end{array}
\right)\\
&&\\
&&
-\frac{1}{ad-bc}
\left(
\begin{array}{cc}
d & -b \\
-c  & a
\end{array}
\right)
\left(
\begin{array}{cc}
0\\
2
\end{array}
\right)\\
&&\\
&&
-\frac{1}{131}
\left(
\begin{array}{cc}
13 & 1 \\
-1  & 10
\end{array}
\right)
\left(
\begin{array}{cc}
0\\
2
\end{array}
\right)
\end{eqnarray*}

Enkel matrisemultiplikasjon gir $(13 \times 0) + (1 \times 2) = 2$ og
$(-1 \times 0) + (10 \times 2) = 20$, vi multipliserer videre med
faktoren $-\frac{1}{131}$ og ender opp med at
\begin{equation*}
\left(
\begin{array}{cc}
\Delta y_{1,0}\\
\Delta y_{2,0}
\end{array}
\right)
= -
\left(
\begin{array}{c}
2\\
20
\end{array}
\right)
\end{equation*}

Det aller siste vi må gjøre er å regne ut de nye verdiene $y_{1,1}$ og
$y_{2,1}$:
\begin{eqnarray*}
y_{1,1} = y_{1,0} + \Delta y_{1,0} = 2 - 0,02 & = &
\underline{\underline{1,98}}\\
y_{2,1} = y_{2,0} + \Delta y_{2,0} = 0 - 0,15 & = &
\underline{\underline{-0,15}}
\end{eqnarray*}

\end{document}
