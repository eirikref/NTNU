\documentclass[norsk, a4paper, 12pt, twoside, titlepage]{article}
\usepackage{babel, alltt, amsmath, float, epsfig, enumerate,
  graphicx, longtable, subfigure, amsfonts, tabularx, boxedminipage,
  fancyheadings}
\usepackage[utf8]{inputenc}
\usepackage[T1]{fontenc}
\usepackage{makeidx}
\usepackage{times}

\pagestyle{fancyplain}

\renewcommand{\sectionmark}[1]%
	     {\markboth{#1}{}}

\lhead[\fancyplain{}{\bfseries\thepage}]%
      {\fancyplain{}{\bfseries\leftmark}}
\rhead[\fancyplain{}{\bfseries\leftmark}]%
      {\fancyplain{}{\bfseries\thepage}}
\cfoot{}


\makeindex

\title{TFE4100 Kretsteknikk\\Formler og data}
\author{Eirik Refsdal <eirikref@pvv.ntnu.no>}

\begin{document}
\maketitle

\tableofcontents


\newpage
\section{Système International d'Unités}
Det er kun oppgitt enheter som er relevante for kretsteknikk.

\index{Strøm!SI-definisjon}
\index{Ampere!SI-definisjon}
\subsection{Grunnenheter}
\begin{tabular}{|p{7.5cm}|p{3cm}|p{3cm}|}
\hline
Kvantitet & Enhet & Symbol \\
\hline
Lengde & Meter & m \\
Masse & Kilogram & kg \\
Tid & Sekund & s \\
Elektrisk strøm & Ampere & A \\
Termodynamisk temperatur & Grader kelvin & K \\
Lysintensitet & Candela & cd \\

\hline
\end{tabular}


\index{Energi!SI-definisjon}
\index{Joule!SI-definisjon}
\index{Effekt!SI-definisjon}
\index{Watt!SI-definisjon}
\index{Ladning!SI-definisjon}
\index{Coulomb!SI-definisjon}
\index{Spenning!SI-definisjon}
\index{Volt!SI-definisjon}
\index{Motstand!SI-definisjon}
\index{Ohm!SI-definisjon}
\index{Konduktans!SI-definisjon}
\index{Siemens!SI-definisjon}
\index{Kapasitans!SI-definisjon}
\index{Farad!SI-definisjon}
\index{Induktans!SI-definisjon}
\index{Henry!SI-definisjon}
\subsection{Avledede enheter}
\begin{tabular}{|p{7.5cm}|p{3cm}|p{3cm}|}
\hline
Kvantitet & Enhet & Formel \\
\hline
Frekvens & Hertz (Hz) & $s^{-1}$ \\
Kraft & Newton (N) & $kg \times m/s^{2}$ \\
Energi eller arbeid & Joule (J) & $N \times m$ \\
Kraft/energi/effekt & Watt (W) & $J/s$ \\
Elektrisk ladning & Coulomb (C) & $A \times s$ \\
Elektrisk spenning & Volt (V) & $W/A$ \\
Elektrisk motstand & Ohm ($\Omega$) & $V/A$ \\
Elektrisk konduktans & Siemens (S) & $A/V$ \\
Elektrisk kapasitans & Farad (F) & $C/V$ \\
Magnetisk fluks & Weber (Wb) & $V \times s$ \\
Induktans & Henry (H) & $Wb/A$ \\
\hline
\end{tabular}


\newpage
\section{Tierpotenser}
\begin{tabular}{|p{7.5cm}|p{3cm}|p{3cm}|}
\hline
Prefiks & Symbol & Potens \\
\hline
atto & a & $10^{-18}$ \\
femto & f & $10^{-15}$ \\
pico & p & $10^{-12}$ \\
nano & n & $10^{-9}$ \\
mikro & $\mu$ & $10^{-6}$ \\
milli & m & $10^{-3}$ \\
centi & c & $10^{-2}$ \\
deka & d & $10^{-1}$ \\
desi & da & $10$ \\
hekto & h & $10^{2}$ \\
kilo & k & $10^{3}$ \\
mega & M & $10^{6}$ \\
giga & G & $10^{9}$ \\
tera & T & $10^{12}$ \\
\hline
\end{tabular}


\newpage
\section{Viktige tall}
\index{Elementærladning}
\begin{tabular}{|p{8.5cm}|p{5cm}|}
\hline
Beskrivelse & Verdi \\
\hline
Elementærladningen & $\pm 1.6022 \times 10^{-19} C$ \\
\hline
\end{tabular}


\newpage
\section{Definisjoner}
\index{Seriekobling!definisjon}
\subsection{Seriekobling}
To eller flere kretselementer sies å være koblet i serie dersom samme
strøm går gjennom hvert av dem.

\index{Parallellkobling!definisjon}
\subsection{Parallellkobling}
To eller flere kretselementer sies å være koblet i parallell dersom
det er samme spenning over hvert av dem.


\newpage
\section{Strøm}
\index{Strøm}
\index{Strøm!definisjon}
\subsection{Definisjon}
\begin{equation}
i = \frac{dq}{dt},
\end{equation}

\begin{eqnarray*}
i & \mbox{---} & \mbox{strøm i ampere (A)} \\
q & \mbox{---} & \mbox{ladning i coulomb (C)} \\
t & \mbox{---} & \mbox{tid i sekunder (s)}
\end{eqnarray*}

\bigskip

\begin{equation}
1 \mbox{ ampere} = \frac{\mbox{1 coulomb}}{\mbox{sekund}}
\end{equation}

\index{Strøm!Kirchoffs lov}
\index{Kirchhoff!strømlov}
\subsection{Kirchhoffs strømlov}
Summen av strømmer inn og ut av en node er lik null.
\begin{equation}
\sum_{n=1}^{N}i_{n} = 0
\end{equation}

\index{Strøm!Ohms lov}
\index{Ohms lov!strøm}
\subsection{Ohms lov}
\begin{equation}
i = \frac{v}{R}
\end{equation}

\begin{eqnarray*}
i & \mbox{---} & \mbox{strøm i ampere (A)} \\
v & \mbox{---} & \mbox{spenning i volt (V)} \\
R & \mbox{---} & \mbox{motstand i ohm (R)}
\end{eqnarray*}

\index{Strømdeler}
\index{Strøm!deling}
\subsection{Strømdeler}
\begin{equation}
i_{n} = \frac{\frac{1}{R_{n}}}{\frac{1}{R_{1}} + \frac{1}{R_{2}} + \cdots + \frac{1}{R_{n}} + \cdots + \frac{1}{R_{N}}} i_{S}
\end{equation}


\newpage
\section{Spenning}
\index{Spenning}
\index{Spenning!definisjon}
\subsection{Definisjon}
\begin{equation}
v = \frac{dw}{dq},
\end{equation}

\begin{eqnarray*}
v & \mbox{---} & \mbox{spenning i volt (V)} \\
w & \mbox{---} & \mbox{energi i watt (W)} \\
q & \mbox{---} & \mbox{ladning coulomb (C)}
\end{eqnarray*}

\bigskip

\begin{equation}
1 \mbox{ volt} = \frac{\mbox{1 joule}}{\mbox{coulomb}}
\end{equation}

\index{Spenning!Kirchhoffs lov}
\index{Kirchhoff!spenningslov}
\subsection{Kirchhoffs spenningslov}
Summen av spenninger i en lukket krets er lik null.
\begin{equation}
\sum_{n=1}^{N}v_{n} = 0
\end{equation}


\index{Spenning!Ohms lov}
\index{Ohms lov!spenning}
\subsection{Ohms lov}
\begin{equation}
v = iR
\end{equation}

\begin{eqnarray*}
v & \mbox{---} & \mbox{spenning i volt (V)} \\
i & \mbox{---} & \mbox{strøm i ampere (A)} \\
R & \mbox{---} & \mbox{motstand i ohm (R)}
\end{eqnarray*}

\index{Spenningsdeler}
\index{Spenning:deler}
\subsection{Spenningsdeler}
\begin{equation}
v_{n} = \frac{R_{n}}{R_{1} + R_{2} + \cdots + R_{n} + \cdots + R_{N}} v_{S}
\end{equation}




\newpage
\section{Effekt}
\index{Effekt}
\index{Effekt!definisjon}
\subsection{Definisjon}

\begin{equation}
p = \frac{dw}{dt} = \left(\frac{dw}{dq}\right) \left(\frac{dq}{dt}\right) = vi,
\end{equation}

\begin{eqnarray*}
p & \mbox{---} & \mbox{effekt i watt (W)} \\
w & \mbox{---} & \mbox{energi i joule (J)} \\
t & \mbox{---} & \mbox{tid i sekunder (s)} \\
q & \mbox{---} & \mbox{ladning i coulomb (C)} \\
v & \mbox{---} & \mbox{spenning i volt (V)} \\
i & \mbox{---} & \mbox{strøm i ampere (A)}
\end{eqnarray*}

\bigskip

\begin{equation}
1 \mbox{ watt} = \frac{\mbox{1 joule}}{\mbox{sekund}}
\end{equation}

\subsection{Omskrivinger vha. Ohms lov}
\begin{equation}
p = (iR)i = i^{2}R
\end{equation}

\begin{equation}
p = v \left(\frac{v}{R}\right) = \frac{v^{2}}{R}
\end{equation}

\subsection{Uttrykt vha. konduktans}
\begin{equation}
p = \frac{i^{2}}{G}
\end{equation}

\begin{equation}
p = v^{2}G
\end{equation}


\newpage
\section{Motstand}
\index{Motstand}
\index{Motstand!definisjon}
\index{Ohms lov}
\subsection{Definisjon (Ohms lov)}
\begin{equation}
R = \frac{v}{i}
\end{equation}

\begin{eqnarray*}
R & \mbox{---} & \mbox{motstand i ohm (R)} \\
v & \mbox{---} & \mbox{spenning i volt (V)} \\
i & \mbox{---} & \mbox{strøm i ampere (A)}
\end{eqnarray*}

\bigskip

\begin{equation}
1 \mbox{ ohm} = \frac{\mbox{1 volt}}{\mbox{ampere}}
\end{equation}

\index{Motstandsevne}
\subsection{Moststandsevne}
Et materiales motstandsevne er oppgitt med symbolet $\rho$.
For et sylinderformet legeme har vi,
\begin{equation}
R = \frac{l}{\sigma A}
\end{equation}

\begin{eqnarray*}
R & \mbox{---} & \mbox{motstand i ohm ($\omega$)} \\
l & \mbox{---} & \mbox{legemets lengde i meter (m)} \\
\sigma & \mbox{---} & \mbox{legemets ledeevne} \\
A & \mbox{---} & \mbox{legemets tverrsnitt i kvadratmeter ($m^{2}$)}
\end{eqnarray*}

\index{Motstand!i seriekobling}
\index{Seriekobling!beregning av motstand}
\subsection{Motstander i serie}
\begin{equation}
R_{eq} = \sum_{n=1}^{N}R_{n}
\end{equation}

\index{Motstand!i parallellkobling}
\index{Parallellkobling!beregning av motstand}
\subsection{Motstander i parallell}
\begin{equation}
R_{eq} = \frac{1}{\frac{1}{R_{1}} + \frac{1}{R_{2}} + \cdots + \frac{1}{R_{N}}}
\end{equation}



\newpage
\section{Konduktans}
\index{Konduktans}
\index{Konduktans!definisjon}
\subsection{Definisjon}
\begin{equation}
G = \frac{1}{R}
\end{equation}

\begin{eqnarray*}
G & \mbox{---} & \mbox{konduktans i siemens (S)} \\
R & \mbox{---} & \mbox{motstand i ohm (R)} 
\end{eqnarray*}

\index{Ledeevne}
\subsection{Ledeevne}
Et materiales ledeevne er oppgitt ved symbolet $\sigma$.


\newpage
\section{Wheatstone bridge}
\index{Wheatstone bridge}
\subsection{Spenningsforskjell mellom node a og b}

\begin{equation}
v_{ab} = v_{a} - v_{b} = v_{s} \left(\frac{R_{2}}{R_{1} + R_{2}} -
\frac{R_{x}}{R_{3} + R_{x}}\right)
\end{equation}

\subsection{Utregning av $R_{x}$}
Broen er i balanse først når strømmen mellom node a og node b er null,
hvilket vil si at begge nodene har samme spenning/potensiale.

\begin{equation}
R_{x} = \frac{R_{2}}{R_{1}}R_{3}
\end{equation}


\newpage
\index{$\Delta$-til-Y}
\section{$\Delta$-til-Y-ekvivalenter}

\subsection{$\Delta$-til-Y}

\begin{equation}
R_{1} = \frac{R_{b}R_{c}}{R_{a} + R_{b} + R_{c}}
\end{equation}

\begin{equation}
R_{2} = \frac{R_{c}R_{a}}{R_{a} + R_{b} + R_{c}}
\end{equation}

\begin{equation}
R_{3} = \frac{R_{a}R_{b}}{R_{a} + R_{b} + R_{c}}
\end{equation}

\index{Y-til-$\Delta$}
\subsection{Y-til-$\Delta$}

\begin{equation}
R_{a} = \frac{R_{1}R_{2} + R_{2}R_{3} + R_{3}R_{1}}{R_{1}}
\end{equation}

\begin{equation}
R_{b} = \frac{R_{1}R_{2} + R_{2}R_{3} + R_{3}R_{1}}{R_{2}}
\end{equation}

\begin{equation}
R_{c} = \frac{R_{1}R_{2} + R_{2}R_{3} + R_{3}R_{1}}{R_{3}}
\end{equation}


\newpage
\printindex
\end{document}
